\documentclass[xcolor=table]{beamer}

\usepackage[polish]{babel}
\usepackage[utf8]{inputenc}
\usepackage[T1]{fontenc}
\usepackage{listings}
\usepackage{lmodern}
\usepackage{textcomp}


\usetheme[language=polish]%
  {Goddard}

\newcommand{\filepath}{\texttt}
\newcommand{\command}{\texttt}
\newcommand{\email}[1]{\href{mailto:#1}{\texttt{#1}}}
\newcommand{\latexcode}{\texttt}
\newcommand{\parameter}[1]{\textlangle #1\textrangle}


\lstset{basicstyle=\ttfamily,keywordstyle=\color{goddardblue}\bfseries,commentstyle=\color{goddardblue!75}\itshape,columns=flexible}

\rowcolors{1}{goddardblue!50}{goddardblue!30}


\title{Szyfrowanie SMS’ów}
\subtitle{technologia oraz przegląd algorytmów}
\author{Bartłomiej Bułat\\
Tomasz Czarnik\\
Krzysztof Śmiłek\\}


\begin{document}

%==============================
\begin{frame}
  \titlepage
\end{frame}


%==============================
\begin{frame}
  \frametitle{Plan}
  \tableofcontents
\end{frame}


%=============================================================
\section{Wstęp}

\begin{frame}
  \frametitle{Wstęp}
 Krótkie wiadomości tekstowe potocznie zwane SMS (Short Message Service) na dobre zadomowiły się w większej części świata i stanowią dzisiaj jeden z popularniejszych kanałów komunikacji. 
% można tu umieścić komiks z XKCD lub dane z niego w porównaniu do całego Internetu 
\end{frame}

\subsection{Zastosowania}
%==============================
\begin{frame}
  \frametitle{Zastosowania}

Stosowane są m. in. do:
\begin{itemize}
\item Wysyłania krótkich wiadomości i "czatów"
\item Aktywowania usług u operatorów komórkowych
\item Powiadomień od operatorów, banków i innych instytucji
\item Wysyłania bezpiecznych tokenów - haseł
\item Reklam i konkursów SMS
\item Sterowania urządzeniami 
%kamerki szpiegowskie hehe
\end{itemize}
   
\end{frame}

%=============================================================
\section{Technologia}

%=============================================================
\subsection{Krótka historia GSM}
\begin{frame}
  \frametitle{Krótka historia GSM}

lorem ipsum i takie tam

\end{frame}

%=============================================================
\begin{frame}
  \frametitle{Powstanie SMS}

lorem ipsum i takie tam

\end{frame}

%==============================
\subsection{Ograniczenia}
\begin{frame}
  \frametitle{Kodowanie liter i liczba liter}

lorem ipsum i takie tam

\end{frame}
%=============================================================
\begin{frame}
  \frametitle{Problem polskich znaków}

lorem ipsum i takie tam

\end{frame}

%==============================
\begin{frame}
  \frametitle{Wielostronnicowe SMS}

lorem ipsum i takie tam

\end{frame}

%=============================================================
\subsection{Zagrożenia}
\begin{frame}
  \frametitle{Zagrożenia}

\begin{itemize}
\item Sposoby podsłuchu 
%np. zdjęcie urządzenia podsłuchowego
\item Podszywanie się pod numery
\end{itemize}

\end{frame}

%=============================================================
\section{SMS od podszewki}
%=============================================================

\subsection{Modemy}
\begin{frame}
  \frametitle{Telefony i modemy 3G}

lorem ipsum i takie tam

\end{frame}
%=============================================================

\begin{frame}
  \frametitle{Bluetooth DUN}

lorem ipsum i takie tam

\end{frame}
%=============================================================
\subsection{Komendy AT}
\begin{frame}
  \frametitle{Komendy AT}

lorem ipsum i takie tam

\end{frame}
%=============================================================

\begin{frame}
  \frametitle{Zestawy komend}

lorem ipsum i takie tam

\end{frame}
%=============================================================
\begin{frame}
  \frametitle{Komendy służące do obsługi SMS}

lorem ipsum i takie tam

\end{frame}
%=============================================================
\begin{frame}
  \frametitle{Kodowanie znaków}

lorem ipsum i takie tam

\end{frame}

%=============================================================
\section{Szyfrowanie}
%=============================================================

\subsection{Kryteria wyboru}
\begin{frame}
  \frametitle{Kryteria wyboru}

 wymagania, szyfrowanie symetryczne/asymetryczne i dlaczego (przede wszystkim kompresja i ograniczona liczba znaków )

\end{frame}

%=============================================================

\subsection{Przegląd algorytmów}


\begin{frame}
  \frametitle{Szyfr T9}

lorem ipsum i takie tam

\end{frame}

\begin{frame}
  \frametitle{Przykład}

lorem ipsum i takie tam

\end{frame}

%=============================================================
\begin{frame}
  \frametitle{Szyfr cezara}

lorem ipsum i takie tam

\end{frame}

\begin{frame}
  \frametitle{Przykład}

lorem ipsum i takie tam

\end{frame}

%=============================================================
\section{Porównianie}
%=============================================================

\begin{frame}
  \frametitle{Porównanie}
    \begin{itemize}
\item Metoda z użyciem T9, nie wymaga oprogramowania, łatwa w złamaniu
\item Szyf cezara, wyjątkowo prosty i nie narzuca dodatkowych danych
\end{itemize}

\end{frame}

%=============================================================
\section{Podsumowanie}

\begin{frame}
  \frametitle{Podsumowanie}
Spośród omawianych przez nas metod każda ma zalety, ale też pewne słabościi.\\[\baselineskip] Przedstawione technologie i sposoby pozwalają łatwo wprowadzić szyfry i zaimplementować je np. w programie komputerowym korzystającym z modemu 3G lub działającym bezpośrednio na urządzeniu (np. pod systemem Android).

\end{frame}

\end{document}
